\documentclass[12pt]{article}

\include{preamble}

\newtoggle{professormode}
\toggletrue{professormode} %STUDENTS: DELETE or COMMENT this line



\title{MATH 369/650 Fall \the\year{} Homework \#2 INCOMPLETE}

\author{Professor Adam Kapelner} %STUDENTS: write your name here

\iftoggle{professormode}{
\date{Due by email 11:59PM ??, September ??, \the\year{} \\ \vspace{0.5cm} \small (this document last updated \today ~at \currenttime)}
}

\renewcommand{\abstractname}{Instructions and Philosophy}

\begin{document}
\maketitle

\iftoggle{professormode}{
\begin{abstract}
The path to success in this class is to do many problems. Unlike other courses, exclusively doing reading(s) will not help. Coming to lecture is akin to watching workout videos; thinking about and solving problems on your own is the actual ``working out.''  Feel free to \qu{work out} with others; \textbf{I want you to work on this in groups.}

Reading is still \textit{required}. For this homework set, read about one and two sided hypothesis testing, the binomial exact test, the one proportion z test, size, level, Fisher's p-values.

The problems below are color coded: \ingreen{green} problems are considered \textit{easy} and marked \qu{[easy]}; \inorange{yellow} problems are considered \textit{intermediate} and marked \qu{[harder]}, \inred{red} problems are considered \textit{difficult} and marked \qu{[difficult]} and \inpurple{purple} problems are extra credit. The \textit{easy} problems are intended to be ``giveaways'' if you went to class. Do as much as you can of the others; I expect you to at least attempt the \textit{difficult} problems. \qu{[MA]} are for those registered for 621 and extra credit otherwise.

This homework is worth 100 points but the point distribution will not be determined until after the due date. See syllabus for the policy on late homework.

Up to 7 points are given as a bonus if the homework is typed using \LaTeX. Links to instaling \LaTeX~and program for compiling \LaTeX~is found on the syllabus. You are encouraged to use \url{overleaf.com}. If you are handing in homework this way, read the comments in the code; there are two lines to comment out and you should replace my name with yours and write your section. The easiest way to use overleaf is to copy the raw text from hwxx.tex and preamble.tex into two new overleaf tex files with the same name. If you are asked to make drawings, you can take a picture of your handwritten drawing and insert them as figures or leave space using the \qu{$\backslash$vspace} command and draw them in after printing or attach them stapled.

The document is available with spaces for you to write your answers. If not using \LaTeX, print this document and write in your answers. I do not accept homeworks which are \textit{not} on this printout. Keep this first page printed for your records.

\end{abstract}

\thispagestyle{empty}
\vspace{1cm}
NAME: \line(1,0){380}
\clearpage
}




\problem{Here we will review theory testing from a conceptual point of view. For each question, state whether the theory under consideration should become a null hypothesis or alternative hypothesis. If null, also write the alternative; if alternative also write the null.}

\begin{enumerate}

\easysubproblem{A new grand unified theory of physics.}\spc{1.5}

\easysubproblem{The latest conspiracy theory about the president.}\spc{1.5}

\easysubproblem{You are a shareholder in a pharmaceutical company. Your new drug cures cancer.}\spc{1.5}

\hardsubproblem{Assume that no virus in any pandemic in history has spread due to asymptomatic transmission. The virus in the current pandemic spreads through asymptomatic transmission.}\spc{2.5}

\end{enumerate}


\problem{We will revisit the concept of a degenerate point estimator. Let $\Xoneton \iid \bernoulli{\theta}$, our DGP and we are focused on point estimation for $\theta$. We then choose the point estimator $\thetahat_{\text{bad}} = 0.1989$.}

\begin{enumerate}

\intermediatesubproblem{Graph the bias of $\thetahat_{\text{bad}}$ over all $\theta$. Label your axes.}\spc{4}

\intermediatesubproblem{Graph the risk of $\thetahat_{\text{bad}}$ under squared error loss. Label your axes.}\spc{5}

\hardsubproblem{Compare $\thetahat_{\text{bad}}$ to $\Xbar$ using the sup risk under squared error loss. How much better is $\Xbar$? With $n$ getting larger does it get even better?}\spc{3}

\hardsubproblem{Assume that $\theta$ is drawn from $\Theta = [0, 1]$ uniformly. This breaks all of our rules about how $\theta$ is one fixed unknown value. But ignore that rule. This allows you to compare both MSE curves by computing $\int_0^1 MSE(\theta) d\theta$. How much better is $\Xbar$? With $n$ getting larger does it get even better?}\spc{7}



For those of you who are curious, this is a metric used in Bayesian statistics called \qu{average risk}. We won't do any Bayesian concepts in this class. You'll have to take 341 for that. 
\end{enumerate}


\problem{Here we will do a binomial exact test. We want to demonstrate that the iphone users in our class is greater than the national average (which is 52.4\%). Recall that our data was as follows: for $n=20$, the $\thetahathat = 0.65$ where the estimator we chose was the sample proportion.}

\begin{enumerate}

\easysubproblem{Write down $H_a$ then $H_0$.}\spc{2}

\easysubproblem{Declare your $\alpha$ level desired for this test. You do not need to justify it. It is what you are comfortable with.}\spc{0}


%\intermediatesubproblem{Because we want to show something is greater than a point value, it is called a right-tailed test. In any test, we need to find the distribution (or approximate the distribution of) the estimator under the null hypothesis. Because we will reject on the right, why is the most conservative value of $\theta$ to choose when deriving the sampling distribution to be largest value in the null hypothesis region (in this case $\theta = \theta_0 = 0.524$)?}
%
%\inred{This question unfortunately isn't answerable without knowing the concept of the \qu{power function} which we didn't get to until lecture 5. So you can skip it.}
%\spc{3}

\easysubproblem{Regardless of if you understood the previous question or not, what is the exact sampling distribution given the null hypothesis? Marked easy because you can copy from class.}\spc{1}

\easysubproblem{Draw the PMF of the sampling distribution. Label all axes carefully and provide sufficient tick marks. Marked easy because you can copy from class.}\spc{8}

\easysubproblem{Indicate the RET and the rejection region in the above illustration. Use your $\alpha$. Everyone's answer may be different!}\spc{-0.5}

\intermediatesubproblem{Were you able to create a rejection region at your exact level of $\alpha$? Yes / no and why?}\spc{3}

\easysubproblem{Run the test. Write your conclusion in English.}\spc{3}

\end{enumerate}

\problem{Now we will test the same theory (i.e. that the iPhone percentage in our class is different than the national average of 52.4\%), but use the approximate one-proportion z test to do it. Recall that our data was as follows: for $n=20$, the $\thetahathat = 0.65$ where the estimator we chose was the sample proportion.}

\begin{enumerate}

\easysubproblem{Write the alternative and null hypotheses again (from lecture).}\spc{0}

\easysubproblem{Write the asymptotic distribution of our estimator under the null hypothesis which we denote $\thetahat~|~H_0$. Answer in terms of standard error and round it to three decimal places. The answer is in lecture 4. Then illustrate the sampling distribution. Label the x-axis and provide tick marks on the x-axis.}\spc{8}

\easysubproblem{What theorem did you use to get the asymptotic distribution of the estimator? State the conditions of the theorem and the theorem's result.}\spc{2}

\easysubproblem{If we employ this asymptotic distribution of the estimator to test our theory, why is this no longer an \emph{exact test} but instead an \emph{approximate test}?}\spc{2}



\intermediatesubproblem{I want to make an apples-apples comparison with the two-sided binomial exact test from lecture. There $\alpha = 2.3\%$ so I want to use that same Type I error setting here. Compute the retainment region and rejection region (remember $\Theta = \zeroonecl$) and denote these two regions in your illustration in (b). To compute these regions, I'll provide you with the following fact: $\Phi(-2.27) = 2.3\% / 2 = 1.15\%$ where $\Phi$ is the CDF of the standard normal rv. (These are the kind of facts that will be provided to you on exams).}\spc{2}

\easysubproblem{How does the retainment region compare here to the retainment region in the binomial exact test from lecture? Is the normal approximation to the binomial a good approximation in this case?}\spc{2}


\hardsubproblem{Write about a scenario where this approximation to the exactness will provide the wrong outcome. Would practical advice does this scenario teach you?}\spc{5}


\easysubproblem{Why is this test named the \emph{two-sided one-proportion z test}?}\spc{3}

\easysubproblem{Run the test and write your conclusion using an English sentence.}\spc{1}

\easysubproblem{What type of error could you have made?}\spc{0}


\easysubproblem{State the definition of Fisher's p-value.}\spc{3}

\intermediatesubproblem{Find the p-value of our estimate as a function of $\Phi$. Illustrate the p-value in the illustration in (b).}\spc{3}

\easysubproblem{Without computing the p-value explicitly, would it be above or below $\alpha = 2.3\%$? Is the estimate \emph{statistically significant}?}\spc{2}

\end{enumerate}



\end{document}
